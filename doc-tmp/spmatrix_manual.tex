\documentclass[a4paper]{article}

\usepackage[]{subfigure}
\usepackage{url}
\usepackage{alltt}
\usepackage{amsfonts} 
\usepackage{amsmath} 
\usepackage{amssymb}
\usepackage{dcolumn}
\newcolumntype{.}{D{.}{.}{-1}}  % column is centered at '.'
\newcolumntype{d}[1]{D{.}{.}{#1}} % right aligned
\usepackage{times}

\newcommand{\kmax}{\ensuremath{k_{\text{max}}}}
\newcommand{\jmin}{\ensuremath{j_{\text{min}}}}
\newcommand{\jmax}{\ensuremath{j_{\text{max}}}}
\newcommand{\epstr}{\ensuremath{\varepsilon_{\text{tr}}}}
\newcommand{\epsjd}{\ensuremath{\varepsilon}}
\newcommand{\epslin}{\ensuremath{\varepsilon_{\text{lin}}}}
\newcommand{\itmax}{\ensuremath{\mathit{it}_{\text{max}}}}
\newcommand{\itmaxlin}{\ensuremath{\mathit{it}_{\text{max,lin}}}}
\newcommand{\istep}{\ensuremath{i_\text{step}}}
\newcommand{\nofit}{\ensuremath{\mathit{it}}}
\newcommand{\mat}[1]{\ensuremath{\boldsymbol{#1}}}
\newcommand{\vect}[1]{\ensuremath{\mathbf{#1}}}
\newlength{\pyindent} \newlength{\pyminipagewidth}
\setlength{\pyindent}{1.0\parindent}
\setlength{\pyminipagewidth}{\textwidth}\addtolength{\pyminipagewidth}{-\pyindent}
\newenvironment{pycode}{\begin{trivlist}\item\hspace*{\pyindent}\begin{minipage}{\pyminipagewidth}\small\begin{alltt}}
      {\end{alltt}\end{minipage}\end{trivlist}}
\newenvironment{pyinline}{\begin{trivlist}\item\hspace*{\pyindent}\begin{minipage}{\pyminipagewidth}\small\begin{alltt}}
      {\end{alltt}\end{minipage}\end{trivlist}}
\newenvironment{arglist} {\begin{list}{}{\setlength{\leftmargin}{4em}\setlength{\itemsep}{0cm}\setlength{\labelwidth}{3em}}}
  {\end{list}}
\renewcommand{\subfigcapskip}{0pt}
\renewcommand{\subfigbottomskip}{0pt}

\title{PySparse - A sparse linear algebra extension for Python} 
\author{Roman Geus} 

\begin{document}
\maketitle

This document is a portion of a draft of my PhD thesis. Unfortunately
the document contains unresolved references to other parts of my
thesis, which are printed as double question marks. Nevertheless the
document could be useful as a reference for the PySparse package.

\subsection{The PySparse package}
\label{sec:python:pysparse}
%
The PySparse package extends the Python interpreter by a set of sparse
matrix types. PySparse also includes modules that implement
\begin{itemize}
\item iterative methods for solving linear systems of equations,
\item a set of standard preconditioners,
\item an interface to a direct solver for sparse linear systems of
  equations,
\item and the JDSYM eigensolver.
\end{itemize}
%
All these modules are implemented as C extension modules for maximum
performance. In the following sections all modules of PySparse are
described in detail.

%------------------------------------------------------------------------
\subsubsection{The spmatrix module}
\label{sec:python:spmatrix}
%
The \textit{spmatrix} module is the foundation of the PySparse
package. It extends the Python interpreter by three new types named
\textit{ll\_mat}, \textit{csr\_mat} and \textit{sss\_mat}. These types
represent sparse matrices in the LL-, the CSR- and SSS-formats
respectively (cf.\ Appendix \ref{sec:spformat}). For all three
formats, double precision values (C type \texttt{double}) are used to
represent the non-zero entries.

The common way to use the \textit{spmatrix} module is to first build a
matrix in the LL-format. The LL-matrix is manipulated until it has its
final shape and content. Afterwards it may be converted to either the
CSR- or SSS-format, which needs less memory and allows for fast
matrix-vector multiplications.

A \textit{ll\_mat} object can be created from scratch, by reading data
from a file (in MatrixMarket format) or as a result of matrix
operation (as e.g.\ a matrix-matrix multiplication). The
\textit{ll\_mat} object supports manipulating (reading, writing,
add-updating) single entries or sub-matrices.

\textit{csr\_mat} and \textit{sss\_mat} are not constructed directly,
instead they are created by converting \textit{ll\_mat} objects. Once
created, \textit{csr\_mat} and \textit{sss\_mat} objects cannot be
manipulated. Their purpose is to support efficient matrix-vector
multiplications.

%------------------------------------------------------------------------
\paragraph{spmatrix module functions}

\subparagraph{ll\_mat(n, m, sizeHint=1000)} 
%
Creates a \textit{ll\_mat} object, that represents a general, all zero
$m\times n$ matrix. The optional \textit{sizeHint} parameter specifies
the number of non-zero entries for which space is allocated initially.

If the total number of non-zero elements of the final matrix is known
(approximately), this number can be passed as \textit{sizeHint}. This
will avoid costly memory reallocations.

\subparagraph{ll\_mat\_sym(n, sizeHint=1000)} 
%
Creates a \textit{ll\_mat} object, that represents a \emph{symmetric},
all zero $n\times n$ matrix. The optional \textit{sizeHint} parameter
specifies, how much space is initially allocated for the matrix.

\subparagraph{ll\_mat\_from\_mtx(fileName)} 
%
Creates a \textit{ll\_mat} object from a file named \textit{fileName},
which must be in MatrixMarket Coordinate format as described at
\url{http://math.nist.gov/MatrixMarket/formats.html}. Depending on the
file content, either a symmetric or a general sparse matrix is
generated.

\subparagraph{matrixmultiply(A, B)}
%
computes the matrix-matrix multiplication 
\begin{equation*}
  \mat{C} := \mat{A}\mat{B}
\end{equation*}
and returns the result $\mat{C}$ as a new \textit{ll\_mat} object
representing a general sparse matrix. The parameters $\mat{A}$ and
$\mat{B}$ are expected to be objects of type \textit{ll\_mat}.

\subparagraph{dot(A, B)}
%
computes the ``dot-product''
\begin{equation*}
  \mat{C} := \mat{A}^T\mat{B}
\end{equation*}
and returns the result $\mat{C}$ as a new \textit{ll\_mat} object
representing a general sparse matrix. The parameters $\mat{A}$ and
$\mat{B}$ are expected to be objects of type \textit{ll\_mat}.

%------------------------------------------------------------------------
\paragraph{ll\_mat objects}
%
\textit{ll\_mat} objects represent matrices stored in the LL format,
which are described in Appendix~\ref{sec:spformat:ll}.
\textit{ll\_mat} objects come in two flavours: general matrices and
symmetric matrices.  For symmetric matrices only the non-zero entries
in the lower triangle are stored.  Write operations to the strictly
upper triangle are prohibited for the symmetric format.  The
\texttt{issym} attribute of an \textit{ll\_mat} object can be queried
to find out whether or not the symmetric storage format is used.

The entries of a matrix can be accessed conveniently using
two-dimensional array indices\footnote{The standard Python language
  does not know multidimensional indices.  However, thanks to Python's
  clever design, its easy to provide multidimensional indices for
  extension types, without any dirty hacks.
  
  In the Python language, subscripts can be of any type (as it is
  customary for dictionaries). A two-dimensional index can be regarded
  as a 2-tuple (the brackets do not have to be written, so
  \texttt{A[1,2]} is the same as \texttt{A[(1,2)]}). If both tuple
  elements are integers, then a single matrix element is referenced.
  If at least one of the tuple elements is a slice (which is also a
  Python object), then a submatrix is referenced.

Subscripts have to be decoded at runtime. This task includes type
checks, extraction of indices from the 2-tuple, parsing of slice
objects and index bound checks. }.  Following Python conventions,
indices start with 0 and wrap around (so -1 is equivalent to the last
index).

The following code creates an empty 5$\times$5 matrix \texttt{A}, sets
all diagonal elements to their respective row/column index and then
copies the value of \texttt{A[0,0]} to \texttt{A[2,1]}.
\begin{pyinline}
>>> from pysparse import spmatrix
>>> A = spmatrix.ll_mat(5, 5)
>>> for i in range(5):
...     A[i,i] = i+1
>>> A[2,1] = A[0,0]
>>> print A
ll_mat(general, [5,5], [(0,0): 1, (1,1): 2, (2,1): 1,
(2,2): 3, (3,3): 4, (4,4): 5])
\end{pyinline}

\noindent The Python slice notation can be used to conveniently access
sub-matrices.
\begin{pyinline}
>>> print A[:2,:]     # the first two rows
ll_mat(general, [2,5], [(0,0): 1, (1,1): 2])
>>> print A[:,2:5]    # columns 2 to 4
ll_mat(general, [5,3], [(2,0): 3, (3,1): 4, (4,2): 5])
>>> print A[1:3,2:5]  # submatrix 
                      # starting at row 1 col 2,
                      # ending at row 2 col 4
ll_mat(general, [2,3], [(1,0): 3])
\end{pyinline}
The slice operator always returns a new \textit{ll\_mat} object,
containing a copy of the selected submatrix.

Write operations to slices are also possible:
\begin{pyinline}
>>> B = ll_mat(2, 2)           # create 2-by-2
>>> B[0,0] = -1; B[1,1] = -1   # diagonal matrix
>>> A[:2,:2] = B               # assign it to upper
>>>                            # diagonal block of A
>>> print A
ll_mat(general, [5,5], [(0,0): -1, (1,1): -1, (2,1): 1, 
(2,2): 3, (3,3): 4, (4,4): 5])
\end{pyinline}

%------------------------------------------------------------------------
\paragraph{ll\_mat object attributes}
%

\subparagraph{A.shape}
%
Returns a 2-tuple containing the shape of the matrix $\mat{A}$, i.e.\ 
the number of rows and columns.

\subparagraph{A.nnz}
%
Returns the number of non-zero entries stored in matrix $\mat{A}$. If
$\mat{A}$ is stored in the symmetric format, only the number of
non-zero entries in the lower triangle (including the diagonal) are
returned.

\subparagraph{A.issym}
%
Returns true (a non-zero integer) if matrix $\mat{A}$ is stored in the
symmetric LL format, i.e. only the non-zero entries in the lower
triangle are stored. Returns false (zero) if matrix $\mat{A}$ is
stored in the general LL format.

\paragraph{ll\_mat object methods}
%

\subparagraph{A.to\_csr()}
%
Returns a newly allocated \textit{csr\_mat} object, which results from
converting matrix $\mat{A}$.

\subparagraph{A.to\_sss()}
%
Returns a newly allocated \textit{sss\_mat} object, which results from
converting matrix $\mat{A}$. This function works for \textit{ll\_mat}
objects in both the symmetric and the general format. If $\mat{A}$ is
stored in the general format, only the entries in the lower triangle
are used for the conversion. No check, whether $\mat{A}$ is symmetric,
is performed.

\subparagraph{A.export\_mtx(fileName, precision=6)}
%
Exports the matrix $\mat{A}$ to file named \textit{fileName}. The
matrix is stored in MatrixMarket Coordinate format as described at
\url{http://math.nist.gov/MatrixMarket/formats.html}. Depending on the
properties of \textit{ll\_mat} object $\mat{A}$ the generated file
either uses the symmetric or a general MatrixMarket Coordinate format.
The optional parameter \textit{precision} specifies the number of
decimal digits that are used to express the non-zero entries in the
output file.

\subparagraph{A.shift(sigma, M)}
%
Performs a daxpy-like operation on matrix $\mat{A}$,
\begin{equation*}
  \mat{A} \leftarrow \mat{A} + \sigma \mat{M}.  
\end{equation*}
The parameter $\sigma$ is expected to be a Python Float object. The
parameter $\mat{M}$ is expected to an object of type \textit{ll\_mat}.

\subparagraph{A.copy()}
%
Returns a new \textit{ll\_mat} object, that represents a copy of the
\textit{ll\_mat} object $\mat{A}$. So,
\begin{pyinline}
>>> B = A.copy()
\end{pyinline}
is equivalent to
\begin{pyinline}
>>> B = A[:,:]
\end{pyinline}
On the other hand
\begin{pyinline}
>>> B = A.copy()
\end{pyinline}
is \emph{not} the same as
\begin{pyinline}
>>> B = A
\end{pyinline}
The latter version only returns a reference to the same object and
assigns it to \texttt{B}. Subsequent changes to \texttt{A} will
therefore also be visible in \texttt{B}.

\subparagraph{A.update\_add\_mask(B, ind0, ind1, mask0, mask1)}
%
This method is provided for efficiently assembling global finite
element matrices. The method adds the matrix $\mat{B}$ to entries of
matrix $\mat{A}$. The indices of the entries to be updated are
specified by \textit{ind0} and \textit{ind1}. The individual updates
are enabled or disabled using the \textit{mask0} and \textit{mask1}
arrays.

The operation is equivalent to the following Python code:
\begin{pyinline}
for i in range(len(ind0)):
    for j in range(len(ind1)):
        if mask0[i] and mask1[j]:
            A[ind0[i],ind1[j]] += B[i,j]
\end{pyinline}
%
All five parameters are NumPy arrays. $\mat{B}$ is an array of rank
two. The four remaining parameters are rank-1 arrays. Their length
corresponds to either the number of rows or the number of columns of
$\mat{B}$.

This method is not supported for \textit{ll\_mat} objects of symmetric
type, since it would generally result in an non-symmetric matrix.
\textit{update\_add\_mask\_sym} must be used in that case.  Attempting
to call this method using a \textit{ll\_mat} object of symmetric type
will raise an exception.

\subparagraph{A.update\_add\_mask\_sym(B, ind, mask)}
%
This method is provided for efficiently assembling symmetric global
finite element matrices. The method adds the matrix $\mat{B}$ to
entries of matrix $\mat{A}$. The indices of the entries to be updated
are specified by \textit{ind}. The individual updates are enabled or
disabled using the \textit{mask} array.

The operation is equivalent to the following Python code:
\begin{pyinline}
for i in range(len(ind)):
    for j in range(len(ind)):
        if mask[i]:
            A[ind[i],ind[j]] += B[i,j]
\end{pyinline}
%
The three parameters are all NumPy arrays. $\mat{B}$ is an array of
rank two representing a square matrix. The four remaining parameters
are rank-1 arrays. Their length corresponds to the order of matrix
$\mat{B}$.

%------------------------------------------------------------------------
\paragraph{csr\_mat and sss\_mat objects}
%
\textit{csr\_mat} objects represent matrices stored in the CSR format,
which are described in Appendix~\ref{sec:spformat:csr}.
\textit{sss\_mat} objects represent matrices stored in the SSS format
(c.f.\ Appendix~\ref{sec:spformat:sss}).  The only way to create a
\textit{csr\_mat} or a \textit{sss\_mat} object is by conversion of a
\textit{ll\_mat} object using the \textit{to\_csr()} or the
\textit{to\_sss()} method respectively. The purpose of the
\textit{csr\_mat} and the \textit{to\_sss()} objects is to provide
fast matrix-vector multiplications for sparse matrices. In addition, a
matrix stored in the CSR or SSS format uses less memory than the same
matrix stored in the LL format, since the \textit{link} array is not
needed.

\textit{csr\_mat} and \textit{sss\_mat} objects do not support
two-dimensional indices to access matrix entries or sub-matrices.
Again, their purpose is to provide fast matrix-vector multiplication.

%------------------------------------------------------------------------
\paragraph{csr\_mat and sss\_mat object attributes}

\subparagraph{A.shape}
%
Returns a 2-tuple containing the shape of the matrix $\mat{A}$, i.e.\ 
the number of rows and columns.

\subparagraph{A.nnz}
%
Returns the number of non-zero entries stored in matrix $\mat{A}$. If
$\mat{A}$ is an \textit{sss\_mat} object, the non-zero entries in the
strictly upper triangle are not counted.

%------------------------------------------------------------------------
\paragraph{csr\_mat and sss\_mat object methods}

\subparagraph{A.matvec(x, y)}
%
Computes the sparse matrix-vector product 
\begin{equation*}
  \vect{y} \leftarrow \mat{A} \vect{x}.
\end{equation*}
$\vect{x}$ and $\vect{y}$ are two double precision, rank-1 NumPy arrays
of appropriate size.

\subparagraph{A.matvec\_transp(x, y)} 
%
Computes the transposed sparse matrix-vector product
\begin{equation*}
  \vect{y} \leftarrow \mat{A}^T \vect{x}.
\end{equation*}
$\vect{x}$ and $\vect{y}$ are two double precision, rank-1 NumPy arrays
of appropriate size. For \textit{sss\_mat} objects
\textit{matvec\_transp} is equivalent to \textit{matvec}.

%------------------------------------------------------------------------
\paragraph{Example: 2D-Poisson matrix}
%
This section illustrates the use of the \textit{spmatrix} module to
build the well known 2D-Poisson matrix resulting from a $n \times n$
square grid.
\begin{pycode}
def poisson2d(n):
    L = spmatrix.ll_mat(n*n, n*n)
    for i in range(n):
        for j in range(n):
            k = i + n*j
            L[k,k] = 4
            if i > 0:
                L[k,k-1] = -1
            if i < n-1:
                L[k,k+1] = -1
            if j > 0:
                L[k,k-n] = -1
            if j < n-1:
                L[k,k+n] = -1
    return L
\end{pycode} 
Using the symmetric variant of the \textit{ll\_mat} object, this gets
even shorter.
\begin{pycode}
def poisson2d_sym(n):
    L = spmatrix.ll_mat_sym(n*n)
    for i in range(n):
        for j in range(n):
            k = i + n*j
            L[k,k] = 4
            if i > 0:
                L[k,k-1] = -1
            if j > 0:
                L[k,k-n] = -1
    return L
\end{pycode}
To illustrate the use of the slice notation to address sub-matrices,
let's build the 2D Poisson matrix using the diagonal and off-diagonal
blocks.
\begin{pycode}
def poisson2d_sym_blk(n):
    L = spmatrix.ll_mat_sym(n*n)
    I = spmatrix.ll_mat_sym(n)
    P = spmatrix.ll_mat_sym(n)
    for i in range(n):
        I[i,i] = -1
    for i in range(n):
        P[i,i] = 4
        if i > 0: P[i,i-1] = -1
    for i in range(0, n*n, n):
        L[i:i+n,i:i+n] = P
        if i > 0: L[i:i+n,i-n:i] = I
    return L
\end{pycode}

%------------------------------------------------------------------------
\subparagraph{Performance comparison with Matlab}
%
Let's compare the performance of three python codes above with the
following Matlab functions:

The Matlab function \texttt{poisson2d} is equivalent to the Python
function with the same name
\begin{pycode}
function L = poisson2d(n)
  L = sparse(n*n);
  for i = 1:n
    for j = 1:n
      k = i + n*(j-1);
      L(k,k) = 4;
      if i > 1, L(k,k-1) = -1; end
      if i < n, L(k,k+1) = -1; end
      if j > 1, L(k,k-n) = -1; end
      if j < n, L(k,k+n) = -1; end
    end
  end
\end{pycode}
\begin{sloppypar}
  The function \texttt{poisson2d\_blk} is an adaption of the Python
  function \texttt{poisson2d\_sym\_blk} (except for exploiting the
  symmetry, which is not directly supported in Matlab).
\end{sloppypar}
\begin{pycode}
function L = poisson2d_blk(n)
  e = ones(n,1);
  P = spdiags([-e 4*e -e], [-1 0 1], n, n);
  I = -speye(n);
  L = sparse(n*n);
  for i = 1:n:n*n
    L(i:i+n-1,i:i+n-1) = P;
    if i > 1, L(i:i+n-1,i-n:i-1) = I; end
    if i < n*n - n, L(i:i+n-1,i+n:i+2*n-1) = I; end
  end
\end{pycode}
\noindent The function \texttt{poisson2d\_kron} demonstrates one of the most efficient 
ways to generate the 2D Poisson matrix in Matlab.
\begin{pycode}
function L = poisson2d_kron(n)
  e = ones(n,1);
  P = spdiags([-e 2*e -e], [-1 0 1], n, n);
  L = kron(P, speye(n)) + kron(speye(n), P);
\end{pycode}

\begin{table}
  \small
  \begin{center}
    \begin{tabular}{|l|*{4}{d{2}|}}
      \hline
      \multicolumn{1}{|c|}{Function} &
      \multicolumn{1}{c|}{$n=100$} &
      \multicolumn{1}{c|}{$n=300$} &
      \multicolumn{1}{c|}{$n=500$} &
      \multicolumn{1}{c|}{$n=1000$} \\
      \hline
      Python \texttt{poisson2d}           &  0.44 &    4.11 &    11.34 & 45.50 \\
      Python \texttt{poisson2d\_sym}      &  0.26 &    2.34 &     6.55 & 26.33 \\
      Python \texttt{poisson2d\_sym\_blk} &  0.03 &    0.21 &     0.62 &  2.22 \\
      Matlab \texttt{poisson2d}           & 28.19 & 3464.9  & 38859    & \multicolumn{1}{c|}{$\infty$} \\
      Matlab \texttt{poisson2d\_blk}      &  6.85 &  309.20 &1912.1  & \multicolumn{1}{c|}{$\infty$} \\
      Matlab \texttt{poisson2d\_kron}     &  0.21 &    2.05 &   6.23 & 29.96 \\
      \hline
    \end{tabular}
    \caption[Performance comparison of Python and Matlab functions to generate the 2D Poisson matrix]{Performance comparison of Python and Matlab functions to generate the 2D Poisson matrix \\ \small The execution times are given in seconds. Matlab version 6.0 Release 12 was used for these timings.}
    \label{tab:python:spmatrix_matpy}
  \end{center}
\end{table}

The execution times reported in Tab.~\ref{tab:python:spmatrix_matpy}
clearly show, that the Python implementation is superior to the Matlab
implementation. If the fastest versions are compared for both
languages, Python is approximately 10 times faster. Comparing the
straight forward \texttt{poisson2d} versions, one is struck by the
result that, the Matlab function is incredibly slow. The Python
version is more then three orders of magnitude faster! This result
really raises the doubt, whether Matlab's sparse matrix format is
appropriately chosen.

\begin{sloppypar}
  The performance difference between Python's \texttt{poisson2d\_sym}
  and \texttt{poisson2d\_sym\_blk} indicates, that a lot of time is
  spent parsing indices.
\end{sloppypar}

%------------------------------------------------------------------------
\subsubsection{The precon module}
\label{sec:python:precon}
%
The \textit{precon} module provides preconditioners, which can be used
e.g.\ for the iterative methods implemented in the in the
\textit{itsolvers} module or the JDSYM eigensolver (in the
\textit{jdsym} module).

In the PySparse framework, any Python object that has the following
properties can be used as a preconditioner:
\begin{itemize}
\item a \textit{shape} attribute, which returns a 2-tuple describing
  the dimension of the preconditioner,
\item and a \textit{precon} method, that accepts two vectors
  $\vect{x}$ and $\vect{y}$, and applies the preconditioner to
  $\vect{x}$ and stores the result in $\vect{y}$. Both $\vect{x}$ and
  $\vect{y}$ are double precision, rank-1 NumPy arrays of appropriate
  size.
\end{itemize}

The \textit{precon} module implements two new object types
\textit{jacobi} and \textit{ssor}, representing Jacobi and the SSOR
preconditioners as described in Sections~\ref{sec:precon:jacobi}
and~\ref{sec:precon:ssor}.

%------------------------------------------------------------------------
\paragraph{precon module functions}
%

\subparagraph{jacobi(A, omega=1.0, steps=1)}
%
Creates a \textit{jacobi} object, representing the Jacobi
preconditioner. The parameter $\mat{A}$ is the system matrix used for
the Jacobi iteration. The matrix needs to be subscriptable using
two-dimensional indices, so e.g.\ an \textit{ll\_mat} object would
work.  The optional parameter $\omega$, which defaults to $1.0$, is
the weight parameter as described in Section~\ref{sec:precon:jacobi}.
The optional \textit{steps} parameter (defaults to $1$) specifies the
number of iteration steps.

\subparagraph{ssor(A, omega=1.0, steps=1)}
%
Creates a \textit{ssor} object, representing the SSOR preconditioner.
The parameter $\mat{A}$ is the system matrix used for the SSOR
iteration.  The matrix $\mat{A}$ has to be an object of type
\textit{sss\_mat}.  The optional parameter $\omega$, which defaults to
$1.0$, is the relaxation parameter as described in
Section~\ref{sec:precon:jacobi}. The optional \textit{steps} parameter
(defaults to $1$) specifies the number of iteration steps.

%------------------------------------------------------------------------
\paragraph{jacobi and ssor objects}
%
Both \textit{jacobi} and \textit{ssor} objects provide the
\textit{shape} attribute and the \textit{precon} method, that every
preconditioner object in the PySparse framework must implement.  Apart
from that, there is nothing noteworthy to say about these objects.

%------------------------------------------------------------------------
\paragraph{Example: diagonal preconditioner}
%
The diagonal preconditioner is just a special case of the Jacobi
preconditioner, with $\omega=1.0$ and $\text{steps}=1$, which happen
to be the default values of these parameters.

It is however easy to implement the diagonal preconditioner using a
Python class:
\begin{pycode}
class diag_prec:
    def __init__(self, A):
        self.shape = A.shape
        n = self.shape[0]
        self.dinv = numpy.zeros(n, 'd')
        for i in xrange(n):
            self.dinv[i] = 1.0 / A[i,i]
    def precon(self, x, y):
        numpy.multiply(x, self.dinv, y)
\end{pycode}
So,
\begin{pyinline}
>>> D1 = precon.jacobi(A, 1.0, 1)
\end{pyinline}
and
\begin{pyinline}
>>> D2 = diag_prec(A)
\end{pyinline}
yield functionally equivalent preconditioners. \texttt{D1} is probably
faster than \texttt{D2}, because it is fully implemented in C.

%------------------------------------------------------------------------
\subsubsection{The itsolvers module}
\label{sec:python:itsolvers}
%
The \textit{itsolvers} module provides a set of iterative methods for
solving linear systems of equations.

The iterative methods are callable like ordinary Python functions. All
these functions expect the same parameter list, and all function
return values also follow a common standard.

Any user-defined iterative solvers should also follow these
conventions, since other PySparse modules rely on them (e.g.\ the
\textit{jdsym} module)

%------------------------------------------------------------------------
\paragraph{Parameter list}
%
Let's illustrate the calling conventions, using the PCG method
defined as \mbox{\textit{info}, \textit{iter}, \textit{relres} =
  \textit{pcg}($\mat{A}$, $\vect{b}$, $\vect{x}$, \textit{tol},
  \textit{maxit}, $\mat{K}$)}.
%
\begin{arglist}
\item[$\mat{A}$] The parameter $\mat{A}$ represents the coefficient
  matrix of the linear system of equations.  $\mat{A}$ must provide
  the \textit{shape} attribute and the \textit{matvec} and
  \textit{matvec\_transp} methods for multiplying with a vector.
\item[$\vect{b}$] The parameter $\vect{b}$, representing the
  right-hand-side of the linear system, is a rank-1 NumPy array.
\item[$\vect{x}$] The parameter $\vect{x}$ is also a rank-1 NumPy
array. On input, $\vect{x}$ holds the initial guess. On output,
$\vect{x}$ holds the approximate solution of the linear system.
\item[\textit{tol}] The \textit{tol} parameter is a float value
  representing the requested error tolerance. The exact meaning of
  this parameter depends on the actual iterative solver.
\item[\textit{maxit}] The \textit{maxit} parameter is an integer that
  specifies the maximum number of iterations to be executed.
\item[$\mat{K}$] The \emph{optional} $\mat{K}$ parameter represents a
  preconditioner object that supplies the \textit{shape} attribute and
  the \textit{precon} method.
\end{arglist}
The iterative solvers may accept additional parameters, which are
passed as keyword arguments.

%------------------------------------------------------------------------
\paragraph{Return value}
%
All iterative solvers return a tuple with three elements
(\textit{info}, \textit{iter}, \textit{relres}):
\begin{arglist}
\item[\textit{info}] is an integer that contains the exit status of
  the iterative solver. \textit{info} has one of the following values
  \begin{list}{}{}
  \item[2] iteration converged, residual is as small as seems
    reasonable on this machine.
  \item[1] iteration converged, $\vect{b} = \vect{0}$, so the exact
    solution is $\vect{x} = \vect{0}$.
    \item[0] iteration converged, relative error appears to be less
      than \textit{tol}.
    \item[-1] iteration not converged, maximum number of iterations
      was reached.
    \item[-2] iteration not converged, the system involving the
      preconditioner was ill-conditioned.
    \item[-3] iteration not converged, an inner product of the form
      $\vect{x}^T \mat{K}^{-1} \vect{x}$ was not positive, so the
      preconditioning matrix $\mat{K}$ does not appear to be positive
      definite.
    \item[-4] iteration not converged, the matrix $\mat{A}$ appears to
      be very ill-conditioned
    \item[-5] iteration not converged, the method stagnated
    \item[-6] iteration not converged, a scalar quantity became too
      small or too large to continue computing
  \end{list}
  So, $\mathit{info} >= 0$ indicates, that $\vect{x}$ holds an
  acceptable solution, and $\mathit{info} < 0$ indicates an error
  condition.

  Note that not all iterative solvers check for all above error
  conditions.
\item[\textit{iter}] holds the of iterations performed.
\item[\textit{relres}] holds relative error of the solution computed
  by the iterative method. What this actually is, depends on the
  actual iterative method used.
\end{arglist}

%------------------------------------------------------------------------
\paragraph{precon module functions}
%
The module functions defined in the \textit{precon} module implement
various iterative methods (PCG, MINRES, QMRS and CGS, cf.\ 
Section~\ref{sec:itsolv}). The parameters and return values conform to
the conventions described above.

\subparagraph{pcg(A, b, x, tol, maxit, K)}
%
The \textit{pcg} function implements the Preconditioned Conjugate
Gradients method.

\subparagraph{minres(A, b, x, tol, maxit, K)}
%
The \textit{minres} function implements the MINRES method.

\subparagraph{qmrs(A, b, x, tol, maxit, K)}
%
The \textit{qmrs} function implements the QMRS method.

\subparagraph{cgs(A, b, x, tol, maxit, K)} The \textit{min} function
implements the CGS method.
%

%------------------------------------------------------------------------
\paragraph{Example: Solving the poisson system}
%
Let's solve the Poisson system
\begin{equation}
  \label{eq:python:1}
  \mat{L} \vect{x} = \vect{1},
\end{equation}
using the PCG method. $\mat{L}$ is the 2D Poisson matrix, introduced
in Section~\ref{sec:python:spmatrix} and $\vect{1}$ is a vector with
all entries equal to one.

\noindent The Python solution for this task looks as follows:
\begin{pycode}
from pysparse import spmatrix, precon, itsolvers
import numpy
n = 300
L = poisson2d_sym_blk(n)
b = numpy.ones(n*n, 'd')
x = numpy.zeros(n*n, 'd')
info, iter, relres = itsolvers.pcg(L.to_sss(), b, x, 1e-12, 2000)
\end{pycode}
The code makes use of the Python function \texttt{poisson2d\_sym\_blk},
which was defined in Section~\ref{sec:python:spmatrix}.

\noindent Incorporating e.g.\ a SSOR preconditioner is straight-forward:
\begin{pycode}
from pysparse import spmatrix, precon, itsolvers
import numpy
n = 300
L = poisson2d_sym_blk(n)
b = numpy.ones(n*n, 'd')
x = numpy.zeros(n*n, 'd')
S = L.to_sss()
Kssor = precon.ssor(S)
info, iter, relres = itsolvers.pcg(S, b, x, 1e-12, 2000, Kssor)
\end{pycode}

\noindent The Matlab solution (without preconditioner) may look as follows:
\begin{pycode}
n = 300;
L = poisson2d_kron(n);
[x,flag,relres,iter] = pcg(L, ones(n*n,1), 1e-12, 2000, ...
                           [], [], zeros(n*n,1));
\end{pycode}

%------------------------------------------------------------------------
\subparagraph{Performance comparison with Matlab and native C}
%
To evaluate the performance of the Python implementation we solve the
2D Poisson system \eqref{eq:python:1} using the PCG method. The Python
timings are compared with results of a Matlab and a native C
implementation.

The native C and the Python implementation use the same core
algorithms for PCG method and the matrix-vector multiplication. On the
other hand, C reads the matrix from an external file instead of
building it on the fly. In contrast to the Python implementation, the
native C version does not suffer from the overhead generated by the
runtime argument parsing and calling overhead.

\begin{table}[h]
  \begin{center}
    \begin{tabular}{|l|l|*{3}{d{2}|}}
      \hline
      \multicolumn{1}{|c|}{Function} &
      \multicolumn{1}{c|}{Size} &
      \multicolumn{1}{c|}{$t_\text{constr}$} &
      \multicolumn{1}{c|}{$t_\text{solv}$} &
      \multicolumn{1}{c|}{$t_\text{tot}$} \\
      \hline
      Python   & $n=100$ &  0.03 &   1.12 &   1.15 \\
               & $n=300$ &  0.21 &  49.65 &  49.86 \\
               & $n=500$ &  0.62 & 299.39 & 300.01 \\
               \hline
      native C & $n=100$ &  0.30 &   0.96 &   1.26 \\
               & $n=300$ &  3.14 &  48.38 &  51.52 \\
               & $n=500$ & 10.86 & 288.67 & 299.53 \\
               \hline
      Matlab   & $n=100$ &  0.21 &    8.85 &    9.06 \\
               & $n=300$ &  2.05 &  387.26 &  389.31 \\
               & $n=500$ &  6.23 & 1905.67 & 1911.8  \\
      \hline
    \end{tabular}
    \caption[Performance comparison of Python, Matlab and native C implementations to solve the 2D Poisson system]{Performance comparison of Python, Matlab and native C implementations to solve the linear system \eqref{eq:python:1} without preconditioning\\ \small The execution times are given in seconds. $t_\text{constr}$ is the time for constructing the matrix (or reading it from a file in the case of native C).  $t_\text{solv}$ is the time spent in the PCG solver. $t_\text{tot}$ is the sum of $t_\text{constr}$ and $t_\text{solv}$. Matlab version 6.0 Release 12 was used for these timings.}
    \label{tab:python:perf_itsolvers}
  \end{center}
\end{table}

Tab.~\ref{tab:python:perf_itsolvers} shows the execution times for the
Python, the Matlab and the native C implementation for solving the
linear system \eqref{eq:python:1}. Matlab is not only slower when
building the matrix, also the matrix-vector multiplication seems to be
implemented inefficiently. Considering $t_\text{solv}$, the
performance of Python and native C is comparable.  The Python overhead
is under 4\% for this case.

%------------------------------------------------------------------------
\subsubsection{The jdsym module}
%
The \textit{jdsym} module provides an implementation of the JDSYM
algorithm (cf.\ Algorithm~\ref{alg:jd-sym}), that is conveniently
callable from Python. The module exports a single function called
\textit{jdsym}.

\paragraph{jdsym(A, M, K, kmax, tau, jdtol, itmax, linsolver, **keywords)}
%
Invokes the JDSYM eigenvalue solver (cf.\ Section~\ref{sec:jd-sym}).
JDSYM computes eigenpairs of a generalised matrix eigenvalue problem
of the form
\begin{equation}
  \label{eq:python:2}
  \mat{A} \vect{x} = \lambda \mat{M} \vect{x} 
\end{equation}
or a standard eigenvalue problem of the form
\begin{equation}
  \label{eq:python:3}
  \mat{A} \vect{x} = \lambda \vect{x},
\end{equation}
where $\mat{A}$ is symmetric and $\mat{M}$ is symmetric
positive-definite.

\subparagraph{Arguments}
%
The \textit{jdsym} function has seven mandatory arguments
\begin{arglist}
\item[$\mat{A}$] This parameter represents the matrix
  $\mat{A}$ in \eqref{eq:python:2} or \eqref{eq:python:3}.  $\mat{A}$
  must provide the \textit{shape} attribute and the \textit{matvec}
  and \textit{matvec\_transp} methods.
\item[$\mat{M}$] This parameter represents the matrix $\mat{M}$ in
  \eqref{eq:python:2}.  $\mat{M}$ must provide the \textit{shape}
  attribute and the \textit{matvec} and \textit{matvec\_transp}
  methods. If the standard eigenvalue problem \eqref{eq:python:3} is
  to be solved, the \texttt{None} value can be passed for this
  parameter.
\item[$\mat{K}$] The $\mat{K}$ parameter represents a preconditioner
  object that supplies the \textit{shape} attribute and the
  \textit{precon} method. If no preconditioner is to used, then the
  \texttt{None} value can be passed for this parameter.
\item[\textit{kmax}] is an integer that specifies the number of
  eigenpairs to be computed.
\item[\textit{tau}] is a float value that specifies the target value
  $\tau$. Eigenvalues in the vicinity of $\tau$ will be computed.
\item[\textit{jdtol}] is a float value that specifies the convergence
  tolerance for eigenpairs $(\lambda,\vect{x})$. The converged
  eigenpairs satisfy $\|\mat{A} \vect{x} - \lambda \mat{M}
  \vect{x}\|_2 <{}$ \textit{jdtol}.
\item[\textit{itmax}] is an integer that specifies the maximum number
  of Jacobi-Davidson iterations to undertake.
\item[\textit{linsolver}] is a function that implements an iterative
  method for solving linear systems of equations. The function
  \textit{linsolver} is required to conform to the standards mentioned
  in Section~\ref{sec:python:itsolvers}.
\end{arglist}

\noindent The remaining (optional) arguments are specified using keyword
arguments:
\begin{arglist}
\item[\textit{jmax}] is an integer that specifies the maximum
  dimension of the search subspace. (default: 25)
\item[\textit{jmin}] is an integer that specifies dimension of the
  search subspace after a restart. (default: 10)
\item[\textit{blksize}] is an integer that specifies the block size
  used in the JDSYM algorithm. (default: 1)
\item[\textit{blkwise}] is an integer that affects the convergence
  criterion if \textit{blksize}${}>1$ (cf.\ 
  Section~\ref{sec:jdsym:block}). (default: 0)
\item[\textit{V0}] is NumPy array of rank one or two. It specifies the
  initial search subspace. (default: a randomly generated initial
  search subspace)
\item[\textit{optype}] is an integer specifying the operator type used
  in the correction equation. If \textit{optype}${}=1$, the
  non-symmetric version is used. If \textit{optype}${}=2$, the
  symmetric version is used. See Section~\ref{sec:correq} for more
  information. (default: 2)
\item[\textit{linitmax}] is an integer specifying the maximum number
  steps taken in the inner iteration (iterative linear solver).
  (default: 200)
\item[\textit{eps\_tr}] is a float value setting the tracking
  parameter $\epstr$ described in Section \ref{sec:jdsym:shift}.
  (default: $10^{-3}$)
\item[\textit{toldecay}] is a float value, that influences the dynamic
  adaption of the stopping criterion of the inner iteration.
  \textit{toldecay} corresponds to the value $\gamma$ in
  Section~\ref{sec:jdsym:stopcrit}. (default: 1.5)
\item[\textit{clvl}] is an integer specifying the ``verbosity'' of the
  \textit{jdsym} function. The higher the \textit{clvl} parameter, the
  more output is sent to the standard output. \textit{clvl}${}=0$
  produces no output. (default: 0)
\item[\textit{strategy}] is an integer specifying shifting and sorting
  strategy of JDSYM. \textit{strategy}${}=0$ enables the default JDSYM
  algorithm. \textit{strategy}${}=1$ enables JDSYM to avoid
  convergence to eigenvalues smaller than $\tau$. (default: 0)
\item[\textit{projector}] is used to keep the search subspace and the
  eigenvectors in a certain subspace. The parameter \textit{projector}
  can actually be any Python object, that provides a \textit{shape}
  attribute and a \textit{project} method. The \textit{project} method
  takes a vector (a rank-1 NumPy array) as its sole argument and
  projects that vector in-place. This parameter can be used to
  implement the DIRPROJ and SAUG methods described in
  Sections~\ref{sec:eigprob:direct} and~~\ref{sec:eigprob:simplaug}.
  (Default: no projection)
\end{arglist}

\subparagraph{Return value}
%
\noindent The \textit{jdsym} module function returns a tuple with four elements
(\textit{kconv}, \textit{lambda}, \textit{Q}, \textit{it}):

\begin{arglist}
\item[\textit{kconv}] is an integer that indicates the number of
  converged eigenpairs.
\item[\textit{lambda}] is a rank-1 NumPy array containing the
  converged eigenvalues.
\item[$\mat{Q}$] is a rank-2 NumPy array containing the converged
  eigenvectors. The i-th eigenvector is accessed by \texttt{Q[:,i]}.
\item[\textit{it}] is an integer indicating the number of
  Jacobi-Davidson steps (outer iteration steps) performed.
\end{arglist}

%------------------------------------------------------------------------
\paragraph{Example: Maxwell problem}
The following code illustrates the use of the \textit{jdsym} module.
Two matrices $\mat{A}$ and $\mat{M}$ are read from files. A Jacobi
preconditioner from $\mat{A} - \tau\mat{M}$ is built. Then the JDSYM
eigensolver is called, calculating 5 eigenvalues near 25.0 and the
associated eigenvalues to an accuracy of $10^{-10}$.  We set
\textit{strategy}${}=1$ to avoid convergence to the high-dimensional
null space of ($\mat{A}$, $\mat{M}$).
\begin{pycode}
from pysparse import spmatrix, itsolvers, jdsym, precon

A = spmatrix.ll_mat_from_mtx('edge6x3x5_A.mtx')
M = spmatrix.ll_mat_from_mtx('edge6x3x5_B.mtx')
tau = 25.0

Atau = A.copy()
Atau.shift(-tau, M)
K = precon.jacobi(Atau)

A = A.to_sss(); M = M.to_sss()
k_conv, lmbd, Q, it  = \symbol{92}
      jdsym.jdsym(A, M, K, 5, tau, 1e-10, 150, itsolvers.qmrs,
                  jmin=5, jmax=10, clvl=1, strategy=1)
\end{pycode}
This code takes $33.71$ seconds to compute the five wanted eigenpairs.
A native C version, using the same computational kernels, takes
$35.64$ for the same task. We expected the Python version to be slower
due to the overhead generated when calling the matrix-vector
multiplication and the preconditioner, but surprisingly the Python
code was even a bit faster.

%------------------------------------------------------------------------
\subsubsection{The superlu module}
%
The \textit{superlu} module interfaces the SuperLU library to make it
usable by Python code. SuperLU is a software package written in C,
that is able to compute a $LU$-factorisation of a general
non-symmetric, sparse matrix with partial pivoting.

The \textit{superlu} module exports a single function, called
\textit{factorize}.

%------------------------------------------------------------------------
\paragraph{factorize(A, diag\_pivot\_thresh, drop\_tol, relax, panel\_size, permc\_spec)}
%
The factorise module function computes a $LU$-factorisation of the
matrix $\mat{A}$. All but the first parameter are optional and can be
specified using keyword arguments.
\begin{arglist}
\item[$\mat{A}$] is a \textit{csr\_mat} object that represents the
  matrix to be factorised.
\item[\textit{diag\_pivot\_thresh}] is a float value in the interval
  $[0,1]$ representing the threshold for partial pivoting.
  \textit{diag\_pivot\_thresh}${}=0$ corresponds to no pivoting.
  \textit{diag\_pivot\_thresh}${}=1$ corresponds to partial pivoting.
  (default: $1.0$)
\item[\textit{drop\_tol}] is a float value in the interval $[0,1]$
  representing the drop tolerance parameter.  \textit{drop\_tol}${}=0$
  corresponds to the exact factorisation. \\
  CAUTION: the \textit{drop\_tol} has no effect in the current and all
  older SuperLU releases (versions 2.0 and below). (default: 0.0)
\item[\textit{relax}] is an integer that controls the degree of
  relaxing supernodes. (default: 1)
\item[\textit{panel\_size}] is an integer specifying the maximum number
  of columns that form a panel. (default: 10)
\item[\textit{permc\_spec}] is an integer specifying the matrix
  ordering used for the factorisation:
  \begin{list}{}{}
  \item[0] natural ordering
  \item[1] MMD applied to the structure of $\mat{A}^T\mat{A}$
  \item[2] MMD applied to the structure of $\mat{A}^T + \mat{A}$
  \item[3] COLAMD, approximate minimum degree column ordering
  \end{list}
  (default: 2)
\end{arglist}

The \textit{factorize} function returns an object of type
\textit{superlu\_context}. This object encapsulates the $LU$-factors
computed during the factorisation.

%------------------------------------------------------------------------
\paragraph{superlu\_context object attributes}
%
\subparagraph{shape} 
%
The \textit{shape} attribute, returns a 2-tuple describing the
dimension of the factorised matrix $\mat{A}$.

\subparagraph{nnz}
%
The \textit{nnz} attribute returns an integer holding the total number
of non-zero entries stored in both the $\mat{L}$ and the $\mat{U}$
factors.

%------------------------------------------------------------------------
\paragraph{superlu\_context object methods}
%
\subparagraph{solve(b, x, trans)}
%
The \textit{solve} method accepts two rank-1 NumPy arrays $\vect{b}$
and $\vect{x}$ of appropriate size and assigns the solution of the
linear system
\begin{equation*}
  \mat{A}\vect{x} = \vect{b}
\end{equation*}
to $\vect{x}$. If the optional parameter \textit{trans} is set to
\texttt{'T'}, then the transposed system
\begin{equation*}
  \mat{A}^T\vect{x} = \vect{b}
\end{equation*}
is solved instead.

%------------------------------------------------------------------------
\paragraph{Example: 2D Poisson matrix}
%
Let's now solve the 2D Poisson system
\begin{equation*}
  \mat{L} \vect{x} = \vect{1},
\end{equation*}
using a factorisation. $\mat{L}$ is the 2D Poisson matrix, introduced
in Section~\ref{sec:python:spmatrix} and $\vect{1}$ is a vector with
all one entries.

\noindent The Python solution for this task looks as follows:
\begin{pycode}
from pysparse import spmatrix, superlu
import numpy 
n = 100
L = poisson2d_sym_blk(n)
b = numpy.ones(n*n, 'd')
x = numpy.zeros(n*n, 'd')
LU = superlu.factorize(L.to_csr(), diag_pivot_thresh=0.0)
LU.solve(b, x)
\end{pycode}
The code makes use of the Python function
\texttt{poisson2d\_sym\_blk}, which was defined in
Section~\ref{sec:python:spmatrix}.

\appendix
\include{spformat}


\end{document}



%%% Local Variables: 
%%% mode: latex
%%% TeX-master: "spmatrix_manual"
%%% End: 
